



\title{My title} %% Title of the presentation



\author[1]{\small FirstName1 LastName1}     %%%% the presenter. The square bracket refers to affiliations as will be defined below.

%%%%%% coauthors. Add or delete as necessary
\author[1, 2]{\small FirstName2 LastName2}
\author[2]{\small FirstName3 LastName3}

%%% Affilliations
\affil[1]{\footnotesize University1, Department1, Country1}
\affil[2]{\footnotesize University2, Department2, Country2}



% generate the titling material
\maketitle


%Provide an abstract eblow
\begin{abstract}
My abstract where I cite this paper  \cite{paper1}.
\end{abstract}

% You can use the standard thebibliography if you need a bibliography
% for your abstract, though we'd prefer not to have any such
% information in the abstact. Note that bibtex or bilatex are not
% allowed. Keeping things simple makes it a lot easier to merge the
% abstracts in the end

% \begin{thebibliography}{9}
% \bibitem{mykey} Some text
% \end{thebibliography}


\begin{thebibliography}{1}
% journal article
\bibitem{paper1}
Chvatal, V., Sankoff, D. (1975). Longest common subsequences of TENrandom
  sequences. \textit{Journal of Applied Probability} 14, 306--315.
\end{thebibliography}
